
\begin{MainChapter}{Introduction}
% \begin{MainChapter}[title=Introduction]

%\Section[title=I am a section, label=ssec, no-entry, no-number]
%\Section[no-entry]{I am a section}
%\Section[title=I am a section]

\textbf{This documentation is very much a work in progress. However, uncomment the code \code{sections/back.tex} to show the (very large) tables for Desert's options and features.}

\section{I am a section}

Turns out \qty{5.302}{\joule\per\kilo\watt} looks ok via \ctan{siunitx}.

\subsection{I am a subsection}

\subsubsection{I am a subsubsection}

\paragraph*{I am a paragraph}  % * to suppress numbering
\indexed{keyword} is shorthand for keyword\index{keyword}.
\Indexed{Quantum xenochromodynamics} is characterized by a module defined $M[F]$ over a finite field $F$ alongside a set $D$ of \ac{D1} ligands \cite{yolov6}.
This is \qtyproduct{61 x 61 x 114}{\centi\meter}.
Add a \code{*} when \emph{introducing} terms, for emphasis: \indexed*{starred}.
The style is set with \code{style / indexed-starred} option \footnote{This is a great idea}. See? \ac{CRISPR}. \ac{CRISPR}. Always put footnotes before the punctuation; \ctan{fnpct} can then kern correctly.
Using \ctan{chemformula}, \ch{KCr(SO4)2 * 12 H2O}. \ch{CH+CH}.

Also see \ctan{chemmacros} and \ctan{chemfig}; check it out \Chem{*5(-=--=)}

\begin{ChemFigure}[
    %caption  = Hello ,
    %label      = chem,
]
\chemfig{[:120]NH_2*6(---=O)}
\end{ChemFigure}

% If you want to make your own numbered flaot type,
% see newfloat: https://mirror.math.princeton.edu/pub/CTAN/macros/latex/contrib/newfloat/newfloat.pdf
% \DeclareFloatingEnvironment {StructFormula} [within=chapter, name=structural formula, listname=...] {figure}

(This too: \indexed{space aliens}[aliens!space]).
It is usually computed in the cloud. With \ac{CRISPR}.

\begin{equation}
\begin{tikzcd}
T
\arrow[drr, bend left, "x"]
\arrow[ddr, bend right, "y"]
\arrow[dr, dotted, "{(x,y)}" description] & & \\
& X \times_Z Y \arrow[r, "p"] \arrow[d, "q"]
& X \arrow[d, "f"] \\
& Y \arrow[r, "g"]
& Z
\end{tikzcd}
\end{equation}


Using \ctan{circuitikz}:

\begin{figure}
\begin{circuitikz}
\draw (0,0) to[R=2<\ohm>, i=?, v=84<\volt>] (2,0) --
(2,2) to[V<=84<\volt>] (0,2)
-- (0,0);
\end{circuitikz}
\end{figure}


Quotations can use \ctan{csquotes}, or key--value commands that Desert added:
As Johnson herself once said, \Quote[cite={johnson2022}, punct=.]{life is good}
She also said that in Spanish, \Quote[style=display, lang=spanish, punct=.]{la vida es buena}
Finally, we have the \code{Quotation} environment, which defaults to \code{style=display}.

We recovered \qty{5}{\tera\joule} from the craft. That's \unit{\kilogram \meter \squared \per \ampere \per \second \cubed}.

Do not use mathcal, etc. Instead, refer to: \link{https://mirror.math.princeton.edu/pub/CTAN/macros/unicodetex/latex/unicode-math/unimath-symbols.pdf}[unimath-symbols].
%$\mbfJ \mbfscrH \mbfscrl \mbffrakF \mathhyphen \BbbZ \mscrH \trprime$

\begin{itemize}
    \item These are short bullet points. These are short bullet points. These are short bullet points. See? See? It's ok if it goes over multiple lines.
    \item These are short bullet points.
    \item These are short bullet points.
\end{itemize}

\newpage
\section{And then...}

\begin{itemize}[long]  % long means multiple paragraphs; it has more space
    \item
    This is a bullet point that can contain multiple paragraphs. Please use responsibly.
    This is a bullet point that can contain multiple paragraphs. Please use responsibly.
    
    See? This continued to a second paragraph, and it needs a little more space between items for that reason. Otherwise it's not clear where a paragraph ends and where a bullet point ends.
    
    \item Again this is happening. We want to be nice to LaTeX.
    
    See? This continued to a second paragraph, and it needs a little more space between items for that reason. Otherwise it's not clear where a paragraph ends and where a bullet point ends.
\end{itemize}

\begin{Theorem}[label=mythm, name=Fiveness]
5 = 5
\end{Theorem}

\begin{Proof}  % same as proof
$x = x \forall x$. $\therefore 5 = 5$
\end{Proof}


\begin{align}
    & \tau(a, b) = \frac{ N_{+1}(a,b) - N_{-1}(a,b)}{\textstyle \binom{n}{2}}
    \label{eqn:kappa}\punc{,}\\
    \shortintertext{where}
    & N_z(a,b) = \sum_{i,j;i<j}\indicator \left[ \, \sgn(a_i - a_j) = z \sgn(b_i - b_j) \neq 0 \, \right]
    \punc{.}  % use \punct{.}, \punct{,}, etc., to end equations
\end{align}

Briefly, $\tau$ was calculated on a $\binom{n}{2}$-length vector of the strict lower triangular elements of each mechanistic and phenotypic affinity matrix, which excluded self-comparisons (diagonal elements).
Minkowski distances were multiplied by $-1$ to provide similarity measures. Mechanistic affinity matrices, 1 per variable $v$, were calculated using a Jaccard index -like function generalized to sets with weighted labels.
Let $P_d(a)$ return the predicate--object pairs for compound $a$ via data source $d$ (where $d$ is one of potentially many sources contributing to $v$). Also let $w_d(a, p)$ return the \emph{weight} of the pair $p$ for $a$. Weights reflect magnitude (e.g. \code{PCHEMBL}).
Values were averaged over unique data sources for final values ${J}(a, b)$ of compounds $a$ and $b$:

\begin{align}
    &
    {J} \left( a, b \right) =
    \underset{
        d ; \: P_d(a) \neq \emptyset
        \text{ and } P_d(b) \neq \emptyset
    }{\text{mean}}
    \ \, {J}_d (a, b)  \punc{;} \\
    &
    {J}_d \left(a, b \right) =
    \underset{p \: \in \: P_d(a) \, \cup P_d(b)}{\mean}
    \left(
    \frac{
         w_d(a) \wedge w_d(b)
    }{
        w_d(a) \vee w_d(b)
    }
    \right)\\
    \shortintertext{where}
    &
    u \wedge v = \sqrt{ \log_2(u) \times \log_2(v) } \text{ and}   \\
    &
    u \vee v = \log_2(u) + \log_2(v) - (u \wedge v)
    \punc{,}
\end{align}

% Hint is an admonition defined by \NewExampleAdmonitions
\begin{Hint}[name = An optional title., label=myhint]
Some ``hint". \ac{D1}
Refer to \cref{mythm}.
\end{Hint}

% This is the admonition we defined ourselves:
\begin{Important}[name=Watch out]
Something important.
\end{Important}

% Tables should be typeset with the modern tabularray
% https://ctan.org/pkg/tabularray
% Use one either these:
%   - Table            | floats (TeX can position intelligently, and respects at= )
%   - LongTable        | can cross pages (uses longtblr; ignores at= )

% longfoot: separate line per note
% shortfoot: make the footer into 1 par
%
We can construct a similar table while setting style options by hand
While we're at it, we'll make it longer.
\begin{Table} [
    %at = ht ,  % optional position (like in \begin{figure}[ht] -- here, top)
    2col,   % span both columns (if 2-column doc)
    simple,  % automatically pretty
    theme = shortfoot,
    entry = {Difficulty modifiers},
    %caption = {Difficulty modifiers},  % OR:
    caption+ = \space(concatenate entry + this as caption) ,
    % You can alternatively set entry=* to set it the same as caption
    label = {difficulty},
    note{\dagger} = {A note about some cells, rows, etc.},
    remark{x} = {Some remark about the table.},
] {
    colspec=lrrrrr,
    cell{2-999}{2-6} = {cmd=\ensuremath}  % in 2022 use cell{2-Z}{2-6} = {mode=math}
}
Type & Trivial (\%) & Easy (\%) & Normal (\%) & Hard (\%) & Strenuous (\%)\\
Type 1 \TblrNote{\dagger} & 300 & 200 & 150 & 100 & 75\\
Type 2 & 70 & 60 & 50 & 50 & 40\\
Type 3 & 80 & 75 & 70 & 65 & 60\\
\end{Table}

We can construct a similar table while setting style options by hand.
While we're at it, we'll make it longer.

\begin{LongTable} [
    at = {} ,  % will be ignored because the table can't float
    simple,
    % plain,  % like simple, but no header row
    %theme = longfoot,
    caption = {Difficulty modifiers 2},
    entry,  % AKA 'entry = *': copy the caption
    label = {difficulty2},
    note{\dagger} = {A note about some cells, rows, etc.},
    remark{x} = {Some remark about the table.},
] {
    %colspec=lrrrrr,
    % Better in most cases to use flexible X specifiers with relative widths:
    colspec = { X[1,l] X[1,r] X[1,r] X[1,r] X[1,r] X[1,r] } ,
    rowhead={1},
    hline{1,Z} = 0.2ex ,
    hline{2} = 0.1ex,
    cells = {font=\sffamily\small},
    row{1} = {font={\sffamily\small\bfseries}},
    cell{2-999}{2-6} = {cmd=\ensuremath},
    row{even} = { bg = evenrow },
}
Type & Trivial (\%) & Easy (\%) & Normal (\%) & Hard (\%) & Strenuous (\%)\\
Type 1 \TblrNote{\dagger} & 300 & 200 & 150 & 100 & 75\\
Type 2 & 70 & 60 & 50 & 50 & 40\\
Type 3 & 80 & 75 & 70 & 65 & 60\\
\end{LongTable}



\DezRule  % rule of same width of table, boxes, etc.

\begin{MintedCode} [
    lang=c++,
    label=hi,         % label, entry, and caption are the same as in \begin{tblr}
    entry=Sample~C++,
    caption=This is some C++ code,
]
int i = 5;
\end{MintedCode}

\begin{Figure} [
    entry      = A figure,
    caption.+  = And more. ,
    label     = figgy,
]
\includegraphics[width=0.1\textwidth]{sections/person.png}
\end{Figure}


\end{MainChapter}
