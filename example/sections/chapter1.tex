
\begin{MainChapter}{Introduction}

\Section{I am a section}

\SubSection{I am a subsection}

\SubSubSection{I am a subsubsection}

\Paragraph*{I am a paragraph}  % * to suppress numbering
% \indx{keyword} is shorthand for keyword\index{keyword}
\Indx{Quantum xenochromodynamics} is characterized by a module defined $M[F]$ over a finite field $F$ alongside a set $D$ of \ac{D1} ligands \cite{yolov6}. Add a \term{*} to emphasize: \indx*{starred} (This too: \indx{space aliens}[aliens!space]).
It is usually computed in the cloud. With \ac{CRISPR}.

\section{Some text}

\blockcquote{johnson2022}{Life is good.}

We recovered \qty{5}{\tera\joule} from the craft.

\begin{itemize}
    \item\texttt{Bullet 1}
    \item\texttt{Bullet 2}
    \item\texttt{Bullet 3}
\end{itemize}

\begin{criterion}[label=mythm, name=Fiveness]
5 = 5
\end{criterion}

\begin{proof}
x = x \forall x\\
\therefore 5 = 5
\end{proof}

\begin{BoxNote}
Some technical note.
Refer to \cref{mythm}.
\end{BoxNote}

\begin{BoxExample}[How to use this (optional title)]
This is an example.
\end{BoxExample}

% Tables should be typeset with the modern tabularray
% Use tblr, talltblr, longtblr, etc.
% In 2022+, use booktabs, talltabs, and longtabs

\begin{longtblr}[
    caption = {Difficulty modifiers},
    entry = {Difficulty modifiers},
    label = {difficulty}
]{
    colspec=lrrrrr, % TODO
    rowhead={1},
    note{a} = {Rescue funds are something really important.},
    remark{x} = {Some random remark.},
    cells={font=\sffamily},
    row{1} = {font={\sffamily\bfseries}},
    cell{2-999}{2-6} = {cmd=\ensuremath}  % TODO: in 2022 vr, mode=math and 2-Z
}
\hline
Type & Trivial & Easy & Normal & Hard & Strenuous\\
\hline
Type 1 \TblrNote{a} (\%) & 300 & 200 & 150 & 100 & 75\\
Type 2 (\%) & 70 & 60 & 50 & 50 & 40\\
Type 3 (\%) & 80 & 75 & 70 & 65 & 60\\
\hline
\end{longtblr}


% Your captioning friends:
% - \capx is like \caption but optionally accepts () for a label
%     e.g. \capx(fig:mylabel)[TOC entry]{Caption}
%          \capx{Caption} works too and is equiv. to \capx()[]{Caption}
% - \capkeyval works the same way tabularray's do:
%     \capkeyval {label=fig:mylabel, entry=TOC entry, caption=Caption}
%     (Wrap any values containing commas in {}; e.g. caption={A caption, a really good one}.

% Your figure friends:
% - \begin{figure}, like always (now always centered though)
% - \SinglePanelFigure for simple figures
% - \phantompanel for fake panels with good anchors
% - \subref to refer to the panel
% - \subcaptionbox or the experimental \realpanel for panels with separate graphics files.\

% Multi-panel figure with a single graphics file and big caption at the end.
%\begin{figure}
%    \includegraphics{full_figure.pdf}
%    % These generate proper hyperlink anchors but are not directly shown:
%    \phantompanel(fig:my_panelA)[My first panel (TOC entry, if applicable)]
%    \phantompanel(fig:my_panelB)[My second panel (TOC entry, if applicable)]
%    \capx(fig:my_bigfig)[Figure TOC entry]{
%       Overview of methods.
%       \subref{fig:my_panelA} Description of panel A.
%       \subref{fig:my_panelB} Description of panel B.
%    }
%\end{figure}

%\singlepanelfigure{
%    file=cows.pdf,
%    label=fig:cows,
%    entry={Photo of cows.},
%    caption={This is some extended text.},
%}

%\SinglePanelFigure[width=5in]{
%    file=cows.png,
%    ...
%}


\begin{sourcecode}{
    language=c++,
    label=hi,         % label, entry, and caption are the same as in \begin{tblr}
    entry=Sample~C++,
    caption=This is some C++ code
}
int i = 5;
\end{sourcecode}

\end{MainChapter}
