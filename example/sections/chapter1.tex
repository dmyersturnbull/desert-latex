
\begin{MainChapter}{Introduction}
% \begin{MainChapter}[title=Introduction]

%\Section[title=I am a section, label=ssec, no-entry, no-number]
%\Section[no-entry]{I am a section}
%\Section[title=I am a section]

\section{I am a section}

For 5.302 J kW some reason \qty{5.302}{\joule\per\kilo\watt} looks ok?

\subsection{I am a subsection}

\subsubsection{I am a subsubsection}

\paragraph*{I am a paragraph}  % * to suppress numbering
\indx{keyword} is shorthand for keyword\index{keyword}.
\Indx{Quantum xenochromodynamics} is characterized by a module defined $M[F]$ over a finite field $F$ alongside a set $D$ of \ac{D1} ligands \cite{yolov6}.
This is \qtyproduct{61 x 61 x 114}{\centi\meter}.
Add a \code{*} when \emph{introducing} terms, for emphasis: \indx*{starred}.
The style is set with \code{style / indx-starred} option. \footnote{This is a great idea.} See? \ac{CRISPR}. \ac{CRISPR}.
Using \ctan{chemformula}, \ch{KCr(SO4)2 * 12 H2O}. \ch{CH+CH}.

Also see \ctan{chemmacros} and \ctan{chemfig}; check it out:

\chemfig{*5(-=--=)}

\chemfig{[:120]NH_2*6(---=O)}

Or:
\StructuralFormulaFigure(a compound){[:120]NH_2*6(---=O)}

Or:
\StructuralFormulaFigure[caption=Another compound]{[:120]NH_2*6(---=O)}

% Alternatively, use the chemmacros 'scheme' module:
% \begin{scheme}
% \chemname{\chemfig{*6((-)-(=O)-*6(-(-NH_2)-(-OH)(=[::60]O)))}}{methanol}
% \end{scheme}
\begin{figure}
\schemestart
\chemname{\chemfig{*6((-)-(=O)-*6(-(-NH_2)-(-OH)(=[::60]O)))}}{methanol}
\schemestop
\end{figure}

% If you want to make your own numbered flaot type,
% see newfloat: https://mirror.math.princeton.edu/pub/CTAN/macros/latex/contrib/newfloat/newfloat.pdf
% \DeclareFloatingEnvironment {StructFormula} [within=chapter, name=structural formula, listname=...] {figure}

\begin{Figure}[caption=methanol-ish]
\schemestart
\chemfig{*6((-)-(=O)-*6(-(-NH_2)-(-OH)(=[::60]O)))}
\schemestop
\end{Figure}

\begin{Figure}[caption=Isoborneol Oxidation Reaction To Camphor]
\schemestart
\chemnameinit{*6(-(-)-(<OH)----)}
\chemname{\chemfig{*6(-(-)-(<OH)----)}}{Isoborneol}
\+
\chemname{\chemfig{NaOCl}}{Sodium\\Hypochlorite}
\arrow{->[HOAc][]}
\chemname{\chemfig{*6(-(-)-(=O)----)}}{Camphor}
\+
\chemname{\chemfig{NaCl}}{Sodium\\Chloride}
\+
\chemname{\chemfig{H_2O}}{Water}
\chemnameinit{}  % reset
\schemestop
\end{Figure}

(This too: \indx{space aliens}[aliens!space]).
It is usually computed in the cloud. With \ac{CRISPR}.

\begin{equation}
\begin{tikzcd}
T
\arrow[drr, bend left, "x"]
\arrow[ddr, bend right, "y"]
\arrow[dr, dotted, "{(x,y)}" description] & & \\
& X \times_Z Y \arrow[r, "p"] \arrow[d, "q"]
& X \arrow[d, "f"] \\
& Y \arrow[r, "g"]
& Z
\end{tikzcd}
\end{equation}


Using \ctan{circuitikz}:

\begin{circuitikz}
\draw (0,0) to[R=2<\ohm>, i=?, v=84<\volt>] (2,0) --
(2,2) to[V<=84<\volt>] (0,2)
-- (0,0);
\end{circuitikz}


\textsc{This had better be in smallcaps.}

Quotations can use \ctan{csquotes}, or key--value commands that Desert added:
As Johnson herself once said, \Quote[cite={johnson2022}, punct=.]{life is good}
She also said that in Spanish, \Quote[style=display, lang=spanish, punct=.]{la vida es buena}
Finally, we have the \code{Quotation} environment, which defaults to \code{style=display}.


See? Is this ok?

\unit{\kilogram \meter \squared \per \ampere \per \second \cubed}
We recovered \qty{5}{\tera\joule} from the craft.
\unit{\kilogram \meter \squared \per \ampere \per \second \cubed}

Do not use mathcal, etc. Instead, refer to: \link{https://mirror.math.princeton.edu/pub/CTAN/macros/unicodetex/latex/unicode-math/unimath-symbols.pdf}.
%$\mbfJ \mbfscrH \mbfscrl \mbffrakF \mathhyphen \BbbZ \mscrH \trprime$

\begin{itemize}
    \item These are short bullet points. These are short bullet points. These are short bullet points. See? See? It's ok if it goes over multiple lines.
    \item These are short bullet points.
    \item These are short bullet points.
\end{itemize}

\newpage
\section{And then...}

\begin{itemize}[long]
    \item
    This is a bullet point that can contain multiple paragraphs. Please use responsibly.
    This is a bullet point that can contain multiple paragraphs. Please use responsibly.
    
    See? This continued to a second paragraph, and it needs a little more space between items for that reason. Otherwise it's not clear where a paragraph ends and where a bullet point ends.
    
    \item Again this is happening. We want to be nice to LaTeX.
    
    See? This continued to a second paragraph, and it needs a little more space between items for that reason. Otherwise it's not clear where a paragraph ends and where a bullet point ends.
\end{itemize}

\begin{Theorem}[label=mythm, name=Fiveness]
5 = 5
\end{Theorem}

\begin{proof}
$x = x \forall x$. $\therefore 5 = 5$
\end{proof}


\begin{align}
    & \tau(a, b) = \frac{ N_{+1}(a,b) - N_{-1}(a,b)}{\textstyle \binom{n}{2}}
    \label{eqn:kappa}\punc{,}\\
    \shortintertext{where}
    & N_z(a,b) = \sum_{i,j;i<j}\indicator \left[ \, \sgn(a_i - a_j) = z \sgn(b_i - b_j) \neq 0 \, \right]
    \punc{.}
\end{align}

Briefly, $\tau$ was calculated on a $\binom{n}{2}$-length vector of the strict lower triangular elements of each mechanistic and phenotypic affinity matrix, which excluded self-comparisons (diagonal elements).
Minkowski distances were multiplied by $-1$ to provide similarity measures. Mechanistic affinity matrices, 1 per variable $v$, were calculated using a Jaccard index -like function generalized to sets with weighted labels.
Let $P_d(a)$ return the predicate--object pairs for compound $a$ via data source $d$ (where $d$ is one of potentially many sources contributing to $v$). Also let $w_d(a, p)$ return the \emph{weight} of the pair $p$ for $a$. Weights reflect magnitude (e.g. \code{PCHEMBL}).
Values were averaged over unique data sources for final values ${J}(a, b)$ of compounds $a$ and $b$:

\begin{align}
    &
    {J} \left( a, b \right) =
    \underset{
        d ; \: P_d(a) \neq \emptyset
        \text{ and } P_d(b) \neq \emptyset
    }{\text{mean}}
    \ \, {J}_d (a, b)  \punc{;} \\
    &
    {J}_d \left(a, b \right) =
    \underset{p \: \in \: P_d(a) \, \cup P_d(b)}{\mean}
    \left(
    \frac{
         w_d(a) \wedge w_d(b)
    }{
        w_d(a) \vee w_d(b)
    }
    \right)\\
    \shortintertext{where}
    &
    u \wedge v = \sqrt{ \log_2(u) \times \log_2(v) } \text{ and}   \\
    &
    u \vee v = \log_2(u) + \log_2(v) - (u \wedge v)
    \punc{,}
\end{align}


\begin{Hint}[name = An optional title., label=myhint]
Some ``hint". \ac{D1}
Refer to \cref{mythm}.
\end{Hint}

\begin{Important}[name=Watch out]
Something important.
\end{Important}

% Tables should be typeset with the modern tabularray
% Use one of these:
%   - Table            (no footer)
%   - TallTable        (can have footer, etc.)
%   - LongTable        (can cross pages)
%   - SimpleTable      ('Simple' variants set style assuming 1 header row at top)
%   - SimpleTallTable
%   - SimpleLongTable 
% Environments: Table (booktabs), LongTable (longtabs), TallTable (talltabs)

% longfoot: separate line per note
% shortfoot: make the footer into 1 par
%
\begin{SimpleTallTable} [
    theme = shortfoot,
    caption = {Difficulty modifiers},
    entry = {Difficulty modifiers },
    label = {difficulty},
    note{\dagger} = {A note about some cells, rows, etc.},
    remark{x} = {Some remark about the table.},
] {
    colspec=lrrrrr,
    cell{2-999}{2-6} = {cmd=\ensuremath}  % in 2022 use cell{2-Z}{2-6} = {mode=math}
}
Type & Trivial (\%) & Easy (\%) & Normal (\%) & Hard (\%) & Strenuous (\%)\\
Type 1 \TblrNote{\dagger} & 300 & 200 & 150 & 100 & 75\\
Type 2 & 70 & 60 & 50 & 50 & 40\\
Type 3 & 80 & 75 & 70 & 65 & 60\\
\end{SimpleTallTable}

We can construct a similar table while setting style options by hand
While we're at it, we'll make it longer.
Note that long tables get centered.

\begin{LongTable} [
    theme = longfoot,
    caption = {Difficulty modifiers 2},
    entry = {Difficulty modifiers 2},
    label = {difficulty2},
    note{\dagger} = {A note about some cells, rows, etc.},
    remark{x} = {Some remark about the table.},
] {
    colspec=lrrrrr, 
    rowhead={1},
    hline{1,Z} = 0.2ex ,
    hline{2} = 0.1ex,
    cells = {font=\sffamily\small},
    row{1} = {font={\sffamily\small\bfseries}},
    cell{2-999}{2-6} = {cmd=\ensuremath},
    row{even} = { bg = evenrow },
}
Type & Trivial (\%) & Easy (\%) & Normal (\%) & Hard (\%) & Strenuous (\%)\\
Type 1 \TblrNote{\dagger} & 300 & 200 & 150 & 100 & 75\\
Type 2 & 70 & 60 & 50 & 50 & 40\\
Type 3 & 80 & 75 & 70 & 65 & 60\\
\end{LongTable}


\dezrule  % rule of same width of table, boxes, etc.

\begin{MintedCode} [
    lang=c++,
    label=hi,         % label, entry, and caption are the same as in \begin{tblr}
    entry=Sample~C++,
    caption=This is some C++ code,
]
int i = 5;
\end{MintedCode}

\begin{OnePanelFigure} [
    entry    = A figure,
    caption  = A figure,
    label    = figgy,
    graphicx = {width=4cm},
    file     = sections/person.png,
]
%\includegraphics[width=0.1\textwidth]{sections/person.png}
\end{OnePanelFigure}


\end{MainChapter}
