
\begin{MainChapter}{Introduction}
% \begin{MainChapter}[title=Introduction]

%\Section[title=I am a section, label=ssec, no-entry, no-number]
%\Section[no-entry]{I am a section}
%\Section[title=I am a section]

\section{I am a section}

\subsection{I am a subsection}

\subsubsection{I am a subsubsection}

\paragraph*{I am a paragraph}  % * to suppress numbering
\indx{keyword} is shorthand for keyword\index{keyword}.
\Indx{Quantum xenochromodynamics} is characterized by a module defined $M[F]$ over a finite field $F$ alongside a set $D$ of \ac{D1} ligands \cite{yolov6}.
Add a \code{*} when \emph{introducing} terms, for emphasis: \indx*{starred}.
The style is set with \code{style / indx-starred} option. \footnote{This is a great idea.} See? \ac{CRISPR}. \ac{CRISPR}.

(This too: \indx{space aliens}[aliens!space]).
It is usually computed in the cloud. With \ac{CRISPR}.

\textsc{This had better be in smallcaps.}

\blockcquote{johnson2022}{Life is good.}

We recovered \qty{5}{\tera\joule} from the craft.

\begin{itemize}
    \item These are short bullet points. These are short bullet points. These are short bullet points. See? See? It's ok if it goes over multiple lines.
    \item These are short bullet points.
    \item These are short bullet points.
\end{itemize}

\newpage
\section{And then...}

\begin{itemize}[multipar]
    \item
    This is a bullet point that can contain multiple paragraphs. Please use responsibly.
    This is a bullet point that can contain multiple paragraphs. Please use responsibly.
    
    See? This continued to a second paragraph, and it needs a little more space between items for that reason. Otherwise it's not clear where a paragraph ends and where a bullet point ends.
    
    \item Again this is happening. We want to be nice to LaTex.
    
    See? This continued to a second paragraph, and it needs a little more space between items for that reason. Otherwise it's not clear where a paragraph ends and where a bullet point ends.
\end{itemize}

\begin{Theorem}[label=mythm, name=Fiveness]
5 = 5
\end{Theorem}

\begin{proof}
x = x \forall x\\
\therefore 5 = 5
\end{proof}

\begin{Hint}[name = An optional title., label=fives]
Some ``hint". \ac{D1}
Refer to \cref{mythm}.
\end{Hint}

% Tables should be typeset with the modern tabularray
% Use tblr, talltblr, longtblr, etc.
% In 2022+, use booktabs, talltabs, and longtabs

\begin{LongTable}[
    caption = {Difficulty modifiers},
    entry = {Difficulty modifiers},
    label = {difficulty},
    note{a} = {Rescue funds are something really important.},
    remark{x} = {Some random remark.},
]{
    colspec=lrrrrr, % TODO
    rowhead={1},
    hline{1,Z} = 1pt,
    hline{2} = 0.5pt,
    cells={font=\sffamily},
    row{1} = {font={\sffamily\bfseries}},
    cell{2-999}{2-6} = {cmd=\ensuremath}  % TODO: in 2022 vr, mode=math and 2-Z
}
Type & Trivial & Easy & Normal & Hard & Strenuous\\
Type 1 (\%) \TblrNote{a} & 300 & 200 & 150 & 100 & 75\\
Type 2 (\%) & 70 & 60 & 50 & 50 & 40\\
Type 3 (\%) & 80 & 75 & 70 & 65 & 60\\
\end{LongTable}


\begin{MintedCode} [
    lang=c++,
    label=hi,         % label, entry, and caption are the same as in \begin{tblr}
    entry=Sample~C++,
    caption=This is some C++ code
]
int i = 5;
\end{MintedCode}

\begin{OnePanelFigure} [
    entry    = A figure,
    caption  = A figure,
    label    = figgy,
    graphicx = {width=4cm},
    file     = sections/person.png
]
%\includegraphics[width=0.1\textwidth]{sections/person.png}
\end{OnePanelFigure}


\end{MainChapter}
