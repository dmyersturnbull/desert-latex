
% ---- math operators
\DeclareMathOperator{\indicator} {\Bbbone}
%\DeclareMathOperator{\indicator} {\mbfsansone}
\DeclareMathOperator{\mean} {mean}
\DeclareMathOperator{\std} {std}

% some unicode-math commands:
% \DeclareSymbolFont
% \DeclareMathAlphabet
% \DeclareMathSymbol
% \DeclareMathDelimiter{⟨symbol⟩}{⟨type⟩}{⟨sym. font⟩}{⟨slot⟩}{⟨sym. font⟩}{⟨slot⟩}
% \DeclareMathRadical{⟨symbol⟩}{⟨type⟩}{⟨sym. font⟩}{⟨slot⟩}{⟨sym. font⟩}{⟨slot⟩}
% https://mirror.mwt.me/ctan/macros/unicodetex/latex/unicode-math/unicode-math.pdf

% and one amsmath command:
\DeclareMathOperator{\sgn}{sgn}

%\NewDocumentCommand \Z {} { \ensuremath{\BbbZ} }
%\NewDocumentCommand \R {} { \ensuremath{\BbbR} }
%\NewDocumentCommand \C {} { \ensuremath{\BbbC} }

\NewDocumentCommand \Z {} { \ensuremath{\BbbZ} }
\NewDocumentCommand \R {} { \ensuremath{\BbbR} }
\NewDocumentCommand \C {} { \ensuremath{\BbbC} }


% ---- units
% From SIUnitX
\DeclareSIQualifier \rxprod {catalyst}  % known subscript
\DeclareSIQualifier \rxprod {product}
% ---- nomenclature

% We'll use nomencl and nomgroups
% https://mirrors.mit.edu/CTAN/macros/latex/contrib/nomencl/nomencl.pdf

\NomsSetup { put-conceptual-groups, put-syntactic-groups }

\PutNomGroups
{
    C=Constants,
    M=Sets,
    N=Sequences,
    U=Units,
    Z=Other symbols
}

% \NomText, \NomMath, and \NomUnit wrap around \nomenclature

% The full syntax (for all three) is:
% \NomText (group) [sort-key] {Symbol} {Description} <URL> [value]
%             ^        ^                               ^      ^
% default:    'Z'      group+Symbol                    none   none
% 'Z' is the "other symbols" group defined above ('Z' so it's sorted last)

% Example 1:
\NomText         {der.}     {Derived expression}
\NomMath (Z) [a] {\sqrt{5}} {The square root of five} [$\approx 2.2360679775$]

\NomUnit {M} {Molar} [\unit{\mol\per\liter}]
\NomUnit {W} {Watt} [\unit{\kilogram \meter \squared \per \second \cubed}]
\NomUnit {V} {Volt} [\unit{\kilogram \meter \squared \per \ampere \per \second \cubed}]
\NomText (U) {Å} {Angstrom} [\qty{1E-10}{\meter}]

% sets ("S")
\NomMath (M) [a1] {\Z} {Integers}
\NomMath (M) [a2] {\Z^+} {Positive integers}
\NomMath (M) [a3] {\Z^{\geq 0}} {Nonnegative integers}
\NomMath (M) [a4] {\Z^-} {Negative integers}
\NomMath (M) [b1] {\Z} {Real numbers}
\NomMath (M) [c1] {\C} {Complex numbers}
\NomMath (M) [c0] {\Z_n} {Finite field of $n$ integers}
\NomMath (M) [d0] {\mathbf{D}_n} {Dihedral group of order $n$}


\NomMath (N) {B_n} {Bell number; ways to partition $n$ items}
    <https://oeis.org/A000110>
    [{$1, 1, 2, 5, 15, 52 \ldots$}]

% Constants ("C")
\NomMath (C) {G} {Gravitational constant}
    <\nistconsturl{bg}>
    [\qty{6.67430e-11}{\meter\cubed\per\kilogram\per\second\squared}]

% other symbols ("Z")
\NomMath {\indicator} {Indicator function}
\NomMath {\mathbf{E}} {Expectation (of a random variable)}
\NomMath {\mathbf{Var}} {Variance (of a random variable)}
\NomMath {A \circ B}
    {Morphological opening of $A$ by $B$}
    <https://en.wikipedia.org/wiki/Opening\_(morphology)>


% ---- abbreviations

% Use acro's \DeclareNewAcronym for all abbreviations

\DeclareAcronym {ToD} {long=time of day}  % sets short=ToD 
\DeclareAcronym {DAG} {long=directed acyclic graph}
\DeclareAcronym {US} {
    short=U.S\acdot ,
    long=United States ,
    first-style=short     % only use the short
}
\DeclareAcronym {DoE} {long=\acs{US} Department of Energy}
\DeclareAcronym {wrt} {short=w.r.t\acdot, long=with respect to}
\DeclareAcronym {MAP} {long=\txtforeign{maximum a posteriori}}
\DeclareAcronym {SAR} {long=structure–activity relationship}  % literal en dash
\DeclareAcronym {UFF} {long=Universal Force Field, cite=uff}
\DeclareAcronym {YOLO} {long=You Only Look Once, cite={yolov6,yolov1}}
\DeclareAcronym {CRISPR} {
    short=CRISPR ,
    long=clustered regularly interspaced short palindromic repeats ,
}
\DeclareAcronym {MOA} {
    long=mechanism of action ,
	long-plural=mechanisms of action ,
}
\DeclareAcronym {MdoA} {
    long=mode of action ,
	long-plural=modes of action ,
	extra={Common abbreviation: \term{MoA}} ,
}
\DeclareAcronym {etc} {
    short = etc\acdot ,
    long = et cetera ,
    format = \textit ,
    first-style = long ,
    plural =
}

\DeclareAcronym {D1} {
    short=D\txtsub{1} ,
    short-acc=D1 ,
    pdfstring=D1 ,
    long=dopamine 1 receptor ,
}
% accsup with alt variant:
\DeclareAcronym {GABA-AR} {
    short=GABA\txtsub{A}R ,
    short-acc=GABA-AR ,
    pdfstring=GABA-AR ,
    long=GABA\txtsub{A} ionotropic receptor ,
    alt=GABA\txtsub{A} ,
    alt-acc=GABA-A ,
    long-acc=GABA-A ionotropic receptor ,
}

\DeclareAcronym {alpha-AR} {long=α-adrenergic receptor}

\DeclareAcronym {sx} {
    short=$\sqrt{\langle S . X \rangle }$ ,
    short-acc=sqrt(<S.X>) ,
    pdfstring=sqrt(<S.X>) ,
    long=something-something ,
}


\DeclareAcronym {CTAN} {
    long=Comprehensive TeX Archive Network ,
}

\DeclareAcronym {DOI} {
    long=Document Object Identifier ,
}

\DeclareAcronym {URL} {
    long=Uniform Resource Locator ,
}

\DeclareAcronym {IEEE} {
    long=Institute of Electrical and Electronics Engineers ,
}

\DeclareAcronym {XMP} {
    long=Extensible Metadata Platform ,
    extra=standard ,
}
\DeclareAcronym {ArXiV} {
    long=arXiv ,
    extra=e-Print archive ,
}

\DeclareAcronym {BioRxiv} {
    long=BioRxiv ,
    extra=e-Print archive ,
}
